\subsubsection{Dynamic Movement Primitives}\label{sec::DMP}
The DMP \cite{SelfStudy} is developed to capture the fundamental dynamics of a motion, including the movement path of a robot's arm and the necessary forces to manipulate it. The DMP represents the movement as a second-order system, specifically a spring-mass-damper system and some force term. Equation \ref{eq:dmp:dmp_eq} refers to this system and describes the dynamics of the DMP.

\begin{align}
    \tau \Dot{z} &= \alpha_y(\beta_y(g-y) - z) + f(x) \notag \\
    \tau \Dot{y} &= z
    \label{eq:dmp:dmp_eq}
\end{align}

Where $\tau$ is a time constant that controls the duration of the trajectory, which makes it possible to speed or slow the trajectory down.
The DMP is built up around a second-order system represented by a spring-mass-damper system as follows. 

\begin{align*}
    \Dot{z} &= \alpha_y(\beta_y(g-y) - z)\\
    \Dot{y} &= z
\end{align*}

The values of $ \alpha_y $ and $ \beta_y$ are both positive constants that are carefully selected to ensure that the system is critically-damped. A common guideline for choosing these parameters is to set $ \beta_y$ equal to one-fourth of $ \alpha_y $.

The force term in equation \ref{eq:dmp:dmp_eq}, works to approximate any given trajectory(set of force changes over time that shapes the trajectory). The force term is represented as a weighted mixture of Gaussian basis functions

\begin{align}
    f(x) &= \frac{\sum_N^{i=1} w_i \phi_i(x)}{\sum_N^{i=1} \phi_i(x)} x(g-y_0)
    \label{eq:dmo:force_func}
\end{align}

Each $\phi_i(x)$ is a Gaussian function and $N$ is the number of weighted functions. These basis functions are combined in a weighted sum to produce a smooth and continuous trajectory that can be used to guide the movement of a robotic arm or other types of system. Generating a trajectory with a higher number of basis functions $N$ results in a more intricate trajectory, while a lower number of basis functions produces a smoother trajectory. 

Looking at equation \ref{eq:dmo:force_func} the force term depends on $x$ rather than time. $x$ is called the phase of the system and it is defined as first-order exponential decay as seen in equation \ref{eq:dmp:phase_func}:

\begin{align}
    \tau \Dot{x} = -\alpha x
\label{eq:dmp:phase_func}
\end{align}

The phase term in the DMP ensures that the movement converges to a position close to zero at the end of the motion. This causes the force term to become zero at the end of the motion, leaving the system with a spring-mass-damper system that ensures convergence to a goal position regardless of external factors. Therefore, the DMP is capable of reaching a goal position at the end of a motion, and one of its key properties is that it is time-independent, making the system autonomous. The DMP is also robust to disturbance, making it an effective technique for responsive movements in dynamic environments. Finally, the DMP requires only a single demonstration to learn the desired trajectory. The DMP can not be better than the demonstration, but it can be as good as the demonstration.

However, implementing the DMP is possible for both joint space and cartesian space. There are some parameters that need to be tuned like $\alpha_y$, $\beta_y$, $\alpha$, $\tau$, and number of basis function $N$.
In this project, a smooth trajectory is deemed necessary rather than a faster or slower one. Thus, the value of $\tau$ is equivalent to the time of the demo training trajectory. To ensure that the spring-mass-damper system is critically damped, $\alpha_y $ is set to $48.0$, and $\beta_y$ is set to $12.0$. The value of $\alpha$ in equation \ref {eq:dmp:phase_func} is chosen to be $-\ln{0.01}$, resulting in a phase system coverage of $99\% $ when $t=\tau$. To achieve a smoother trajectory, the number of Gaussian basis functions $N$ is selected to be 10.

However, the training and generated trajectory using the DMP in Cartesian space is depicted in figure \ref{fig:Adm:dmp_both}. 

\begin{figure}[H]
  \centering
  \begin{subfigure}[b]{0.45\linewidth}
    \includegraphics[width=\linewidth]{Tables and Images/Imlementing_fig/position_dmp.pdf}
    \caption{Position of DMP for generating extract skills.}
    \label{fig:impl:dmp:pos}
  \end{subfigure}
  \hfill
  \begin{subfigure}[b]{0.45\linewidth}
     \includegraphics[width=\linewidth]{Tables and Images/Imlementing_fig/rotation_dmp.pdf}
    \caption{Orientation of DMP for generating extract skills.}
    \label{fig:impl:dmp:ori}
  \end{subfigure}
  \caption{DMP versus DEMO for skill Down A.}
  \label{fig:Adm:dmp_both}
\end{figure}
\newpage

The aim is to achieve a smoother trajectory obtained as seen in the two figures \ref{fig:impl:dmp:pos} and \ref{fig:impl:dmp:ori}. 
However, if the value of $N$ exceeds 10, the trajectory obtained is identical to the demonstration, while a value lower than 10 leads to oscillation and an imprecise path. Thus, for this particular application, $N$ equals 10 is the most suitable option.
